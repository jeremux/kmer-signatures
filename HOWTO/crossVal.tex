\subsection{Validation croisée à un niveau taxonomique}
\label{graph}

\begin{verbatim}
  $ cd cpp/count_kmer/
\end{verbatim}

\subsubsection{Résultats attendus}

Un fichier au format tabulé avec
\begin{itemize}
  \item La première colonne : la taille des reads de l'apprentissage
  \item la seconde : la taille des reads de l'espace à prédire
  \item et la dernière : le taux de vrais positifs 
\end{itemize}

Ce fichier ce nomme result.log .

\subsubsection{Utilisation}
Les options
\begin{itemize}
		\item \textit{noData}              libérer la mémoire suite au chargement des données,
		\item \textit{kmer}                fichier contenant les patterns de kmers,
		\item \textit{key}                 sur quels séquences compter: cox1,cox2,genomes...,
		\item \textit{root}                racine où on doit établir l’apprentissage,
		\item \textit{learn}               taille de l'apprentissage pour la validation croissée,
		\item \textit{start}               début de la taille du read dans la prédiction pour la validation croisée,
		\item \textit{end}                 fin de la taille du read dans la prédiction pour la validation croisée,
		\item \textit{step}                taille du pas pour les différentes fenêtre pour la prédiction,
		\item \textit{sample}              nombre de séquences à considérer par taxon,
		\item \textit{list}                une liste de taille pour la taille de l'apprentissage
\end{itemize}
~\\


\subsubsection{Exemples}
Utilisation de base voir figure \ref{crossV}

~\\
\begin{figure}[H]
\begin{center}
\begin{verbatim}
$ perl execAndEval.pl --root "../../create_db/Eukaryota__2759" 
                      --start "100" 
                      --end "300" 
                      --step "50" 
                      --list list.txt 
                      --kmer "pattern.txt" 
                      --sample "20"
\end{verbatim}
\end{center}
\caption{\label{crossV}Utilisation du script execAndEval.pl}
\end{figure}
~\\

Si on lance la commande de la figure \ref{crossV} et si list.txt correspond au fichier de la figure \ref{fileList}
alors notre fichier aura la forme de celui présenté en figure \ref{resultlog}

~\\
\begin{figure}[H]
\begin{center}

\begin{verbatim}
$ cat list.txt
-1
1000
200
\end{verbatim}
\end{center}
\caption{\label{fileList}Contenu du fichier list.txt}
\end{figure}
~\\

\begin{figure}[H]
\begin{verbatim}
$ cat list.txt
complete    100	    61.7
complete	  150	    63.1
complete	  200	    64.7
complete	  250	    73.9
complete	  300	    77.3
1000	      100	    68.9
1000	      150	    44.3
1000	      200	    50.7
1000	      250	    66.9
1000	      300	    62.9
200	        100	    91.3
200	        150	    98.4
200	        200	    97.3
200	        250	    98.4
200	        300	    99.0

\end{verbatim}
\begin{center}

\end{center}
\caption{\label{resultlog}Contenu du fichier result.log}
\end{figure}
~\\



\subsection{Génération d'une courbe à partir d'un fichier tabulé}
\label{graph2}

\begin{verbatim}
  $ cd generate_graph
\end{verbatim}
\subsubsection{Prérequis}

Un fichier au format tabulé avec à la première colonne
la taille du read de l'espace d'apprentissage, à la suivante 
la taille du read de l'espace de prédiction et à la dernière colonne
au taux de vrais positifs

\subsubsection{Résultats attendus}

L'outil génère un fichier au format pdf représentant le graphe de vrai positif
en fonction de la taille du read. 


\subsubsection{Exemples}
Utilisation de base voir figure \ref{makgraph}

~\\
\begin{figure}[H]
\begin{center}
\begin{verbatim}
$ perl ./make_graph.pl -in ../cpp/count_kmer/result.log -title eukaryotaComplete
\end{verbatim}
\end{center}
\caption{\label{makgraph}Utilisation du script make\_graph.pl}
\end{figure}
~\\
