\pdfminorversion 7
\pdfobjcompresslevel 3

\PassOptionsToPackage{table}{xcolor}
\documentclass[10pt,a4paper]{article}
%\documentclass[10pt,oneside,noprintercorrection]{article}
\special{papersize=210mm,297mm}


\usepackage[absolute]{textpos} 
\usepackage{pdfpages}
\usepackage[utf8]{inputenc}
\usepackage[T1]{fontenc}
\usepackage{cite}
\usepackage[francais]{babel}
\usepackage[bookmarks=false,colorlinks,linkcolor=blue]{hyperref}
\usepackage[top=4cm,bottom=3cm,left=3cm,right=3cm]{geometry}
\usepackage{graphicx}
\usepackage{wrapfig}
\usepackage{subfig}
\usepackage{eso-pic}
\usepackage{array}
\usepackage{listings}
\usepackage{color}
\usepackage[table]{xcolor}
\usepackage{url}
\usepackage{eurosym}
\usepackage{url}
\usepackage{textcomp}
\usepackage{fancyhdr} 
\usepackage{amsmath}
\usepackage{pict2e}
\usepackage{listings}
\usepackage{setspace}
\usepackage{float}
\usepackage[ruled,vlined,linesnumbered]{algorithm2e}
\usepackage[toc,page]{appendix} 
\usepackage{varwidth}
\onehalfspacing
\lstset{escapeinside={<@}{@>}}
\definecolor{lightgray}{gray}{0.9}


\title{Manuel d'utilisation}

\author{Jérémy Fontaine}


\begin{document}

\newpage
\tableofcontents
\newpage
\listoffigures

\newpage
\section{Introduction}
\vspace{10em}
Le but de ce document est de présenter les différents outils développés, pour permettre à un utilisateur externe aux projet de 
pouvoir l'utiliser. La présentation se fera dans l'ordre chronologique des développements, on commencera tout d'abord par la
construction de la base de données en locale, puis par le comptage en fréquence de kmers et on terminera par la validation croisée.
\newpage

\section{Espace de travail et prérequis}
On suppose que l'espace de travail par défaut est celle qui contient les dossiers présentés en figure  :

~\\
\begin{figure}[H]

\begin{center}
\begin{verbatim}
  $ ls -1
count_kmer/  cpp/  create_db/  data_mining/  docs/  filter/ generate_data/  generate_graph/  
generate_learn/  HOWTO/  qwery/ REQUIRED
\end{verbatim}


\end{center}
\caption{\label{ws}Espace de travail par défaut}
\end{figure}
~\\

\begin{itemize}
  \item qwery/ : Voir section \ref{qwery},
  \item create\_db/: voir section \ref{createdb},
  \item generate\_data/: voir section \ref{generatedata},
  \item cpp: voir section \ref{cpp},
  \item generate\_graph: voir seciton \ref{graph}.
\end{itemize}
\newpage

\section{Construction de la base de données locale}
\subsection{Récupération du Genbank}
\label{qwery}
\subsubsection{Prérequis}

Dans l'espace de travail par défaut se trouve un fichier REQUIRED, qui un
script d'installation de paquet Debian nécessaire  pour l'utilisation des scripts perl, voir figure \ref{installPerl}

~\\
\begin{figure}[H]

\begin{center}
\begin{verbatim}
  $ ls -1
count_kmer/  cpp/  create_db/  data_mining/  docs/  filter/ generate_data/  generate_graph/  
generate_learn/  HOWTO/  qwery/ REQUIRED

  $ sh REQUIRED
\end{verbatim}


\end{center}
\caption{\label{installPerl}Installation des bibliothèque perl}
\end{figure}
~\\

On peut à présent se placer dans le dossier qwery pour effectuer notre requete.

\begin{verbatim}
  $ cd qwery/
  
  $ ls
  qweryNCBI.pl
\end{verbatim}

\subsubsection{Résultats attendus}

On s'attend ici à récupérer un fichier au format Genbank du ncbi\footnote{http://www.ncbi.nlm.nih.gov/} des génomes
mitochondriaux complet.

La requête exécutée par défaut si l'id = 2759 est : 
\begin{verbatim}
  txid2759[Organism:exp] AND (mitochondria[Title] OR mitochondrion[Title] OR mitochondrial[Title]) AND "complete genome"[Title]
\end{verbatim}

\subsubsection{Utilisation}
\paragraph{Les options}
\begin{itemize}
 \item Obligatoires :
\begin{itemize}
 
     \item id: taxid du taxon à récupérer ( -id 2759 )
     \item m: email de l'utilisateur ( -m myemail@mail.com )
    \item out: nom de sortie sans extension ( -out eukaryota )
  \end{itemize}
  \item Optionelles:
  \begin{itemize}
    \item path : dossier où sauvegarder le genbank ( /home/me/bdd/ )
    \item not : liste d'id à ne pas récupérer ( -not 33630,33258 )
    \item more : ajouter des précisions à la requêtes par défaut ( -more "NOT HOMO" )
    \item mine : spécifier sa propre requête ( -mine txid2759[Organism:exp] )
    \item help : Affiche l'aide.
  \end{itemize}
\end{itemize}


\subsubsection{Exemples}

Utilisation basique voir figure \ref{ub}:

~\\
\begin{figure}[H]

\begin{center}
\begin{verbatim}
  $ ls 
  qweryNCBI.pl
  
  $ ./qweryNCBI.pl -id 2759 -m toto@mail.com -out eukaryota
  
  $ ls
  qweryNCBI.pl eukaryota.gb
  \end{verbatim}
\end{center}
\caption{\label{ub}Utilisation minimale de qweryNCBI.pl}
\end{figure}
~\\

\subsection{Création des dossier}
\label{createdb}

\begin{verbatim}
  $ cd create_db
  $ ls 
  bdd/ Eukaryota_krona.html  generateDirectories.pl  
  get_dump_file.sh  get_leaf.sh  install_krona.sh
\end{verbatim}
\subsubsection{Prérequis}

Un fichier genbank est nécessaire pour la création de la base de donnée.
Les requêtes sont par défaut effectuées en locale il est donc d'avoir la structure arborescente du NCBI 
dans des fchiers plats, ces fichiers sont versionnés sur la forge et sont présent dans create\_db/bdd/, mais on peut également les récupérer ou mettre
à jour via le script\textit{ get\_dump\_file.sh }

\begin{verbatim}
  $ sh get_dump_file.sh
  \end{verbatim}



\subsubsection{Résultats attendus}
Le script produit une base de donnée avec une structure arborescente de dossier. Il produit également, la structure
de la base dans un format newick, un fichier xml au format adéquat pour l'outil \textit{krona}, et un script pour le nettoyage de la base.

Le script \textit{install\_krona.sh} est prévu pour l'installation de krona, si on souhaite l'utiliser voir figure \ref{krona}.

Enfin il va générer deux fichier dans le dossier generate\_data pour la génération
des données pour l'étape \ref{generatedata}.
~\\
\begin{figure}[H]

\begin{center}
\begin{verbatim}
14:47:09 [jeremy][create_db]$ ./install_krona.sh 

*********************
Debut install krona
*********************


Cloning into 'krona'...
remote: Counting objects: 788, done.
remote: Compressing objects: 100% (598/598), done.
remote: Total 788 (delta 364), reused 305 (delta 149)
Receiving objects: 100% (788/788), 528.43 KiB | 395 KiB/s, done.
Resolving deltas: 100% (364/364), done.
Creating links...

Installation complete.

To use scripts that rely on NCBI taxonomy, run updateTaxonomy.sh to build the
local taxonomy database.


**************************
Fin install krona, exemple
**************************


*********************
ktImportXML krona_Eukaryota.xml
*********************


14:47:19 [jeremy][create_db]$ 

  \end{verbatim}
\end{center}
\caption{\label{krona}Installation de krona}
\end{figure}
~\\

\subsubsection{Utilisation}
\paragraph{Les options}
\begin{itemize}
 \item Obligatoires :
\begin{itemize}
 
     \item id: taxid du taxon  ( -id 2759 )
     \item gen: fichier genbank ( -gen file.gb )
    \item bound: nombre de sequence pour creer un dossier ( -bound 10 )
  \end{itemize}
  \item Optionelles:
  \begin{itemize}
    \item path : dossier où créer la base de donnée ( /home/me/bdd/ )
    \item time : temps pris pour la création de la base ( -time )
    \item help : Affiche l'aide.
  \end{itemize}
\end{itemize}

\subsubsection{Exemples}

Utilsation de base voir figure \ref{gd}:

~\\
\begin{figure}[H]

\begin{center}
\begin{verbatim}
  $ ls
  bdd/  Eukaryota_krona.html  generateDirectories.pl  get_dump_file.sh  
  get_leaf.sh  install_krona.sh 
  
  $ ls ../generate_data 
  conf  extractGenbank.pl  fillAll_v2.sh
  
  $ ./generateDirectories.pl -id 33630 -gen ../qwery/alveolata.gb -bound 10
  
  $ ls 
  Alveolata__33630/  bdd/  Eukaryota_krona.html  generateDirectories.pl  get_dump_file.sh  
  get_leaf.sh  install_krona.sh  krona_Alveolata.xml  script_clean_Alveolata.sh  
  tree_Alveolata.newick
  
  $ ls ../generate_data 
  conf  extractGenbank.pl  fillAll_v2.sh	generateGenbank_Alveolata.sh  listGenbank.txt
  \end{verbatim}
\end{center}
\caption{\label{gd}Utilisation minimale de generateDirectories.pl}
\end{figure}
~\\



\subsection{Création des dossier}
\label{generatedata}
\begin{verbatim}
  $ cd generate_data
  $ ls
  conf  extractGenbank.pl  fillAll_v2.sh	generateGenbank_Alveolata.sh  listGenbank.txt
\end{verbatim}

\subsubsection{Prérequis}
Le fichier listGenbank, généré grâce au script présenté en \ref{createdb} est nécessaire 
pour la génération des données aux bons endroits. En effet ce fichier contient des listes d'accessions propre
à un dossier.

\subsubsection{Résultats attendus}
On s'attend, après avoir exécuté ce scripts, l'ensemble des données dans la base. C'est à dire
les genbanks correspondants, les génomes, les séquences codantes (selon le fichier conf)...

\subsubsection{Utilisation, exemple}

~\\
\begin{figure}[H]
\begin{center}
\begin{verbatim}
  $ sh generateGenbank_Alveolata.sh 
  \end{verbatim}
\end{center}
\caption{Génération des données}
\end{figure}
~\\


Attention, ce script produit un affichage sur stdout montrant l'évolution du script dans ces trois étapes:

\begin{verbatim}
      Traitement 1 / 3
      Avancement: 100 %
      Traitement 2 / 3 (Génération des donnees aux feuilles)
      Traitement 3/3 (linkage)
\end{verbatim}




\newpage
\section{Fréquence de kmers}
\subsection{Calcul de fréquence en c++}
\label{cpp}

  \begin{verbatim}
  $ cd cpp/count_kmer/
  
  $ ls
  qweryNCBI.pl
\end{verbatim}
Il existe également une version de calcul de fréquence de kmer en C, mais pour
le projet on a opté pour le c++ pour son côté objet. Les versions développés en Perl, Java et C
ont servi de test pour la version c++.

\subsubsection{Utilisation}
Les options
\begin{itemize}
  \item \textit{listFasta}           fichier contenant une liste de chemins vers des fichiers fasta,
		\item \textit{fasta }              fichier fasta,
		\item\textit{wsize}               taille du read,
		\item \textit{noData}              libérer la mémoire suite au chargement des données,
		\item \textit{kmer}                fichier contenant les patterns de kmers,
		\item \textit{output}              nom du fichier de sortie,
		\item \textit{version}             version du programme,
		\item \textit{test}                lance une batterie de tests unitaires,
		\item \textit{intra}               lance un test sur le taxon Intramacronucleata,
		\item \textit{key}                 sur quels séquences compter: cox1,cox2,genomes...,
		\item \textit{root}                racine où on doit établir l'apprenstissage,
		\item \textit{jump}                taille du décalage de la fenetre,
		\item \textit{learn}               taille de l'apprentissage pour la validation croissée,
		\item \textit{start}               debut de la taille du read dans la prédiction pour la validation croissée,
		\item \textit{end}                 fin de la taille du read dans la prédiction pour la validation croissée,
		\item \textit{step}                taille du pas pour les différentes fenetre pour la prédiction,
		\item \textit{sample}              nombre de sequences à considérer par taxon,
		\item \textit{weka}                imprime sur la sortie le format weka,
		\item \textit{help}                affiche l'aide.
\end{itemize}
\subsubsection{Exemples}


~\\
\begin{figure}[H]

Pour produire un fichier de fréquence au format weka avec un fichier fasta voir la figure \ref{freq1}.
Cependant utiliser un fichier simple ne produira pas de classe à prédire pour cela il faut fournir un dossier \ref{freq2}
avec l'architecture établi par le programme présenté en \ref{createdb}.
\begin{center}
\begin{verbatim}
  $ makefile
  
  $ ./count_kmer -f seq.fasta -k pattern.txt -o out.arff -l -1 --weka
  
  $ cat out.arff
  @RELATION freqKmer

  @ATTRIBUTE aaaa NUMERIC
  @ATTRIBUTE aaac NUMERIC
  @ATTRIBUTE aaag NUMERIC
  @ATTRIBUTE aaat NUMERIC
  ...
  @ATTRIBUTE tttg NUMERIC
  @ATTRIBUTE tttt NUMERIC
  @ATTRIBUTE class {null}

  @DATA
  0.25,0,...,null %Datat(0,0)

  \end{verbatim}
\end{center}
\caption{\label{freq1}Produit un fichier weka avec count\_kmer}
\end{figure}
~\\



\newpage
\section{Validation croisée}
\subsection{Validation croisée à un niveau taxonomique}
\label{graph}

\begin{verbatim}
  $ cd cpp/count_kmer/
\end{verbatim}

\subsubsection{Résultats attendus}

Un fichier au format tabulé avec
\begin{itemize}
  \item La première colonne : la taille des reads de l'apprentissage
  \item la seconde : la taille des reads de l'espace à prédire
  \item et la dernière : le taux de vrais positifs 
\end{itemize}

Ce fichier ce nomme result.log .

\subsubsection{Utilisation}
Les options
\begin{itemize}
		\item \textit{noData}              libérer la mémoire suite au chargement des données,
		\item \textit{kmer}                fichier contenant les patterns de kmers,
		\item \textit{key}                 sur quels séquences compter: cox1,cox2,genomes...,
		\item \textit{root}                racine où on doit établir l’apprentissage,
		\item \textit{learn}               taille de l'apprentissage pour la validation croissée,
		\item \textit{start}               début de la taille du read dans la prédiction pour la validation croisée,
		\item \textit{end}                 fin de la taille du read dans la prédiction pour la validation croisée,
		\item \textit{step}                taille du pas pour les différentes fenêtre pour la prédiction,
		\item \textit{sample}              nombre de séquences à considérer par taxon,
		\item \textit{list}                une liste de taille pour la taille de l'apprentissage
\end{itemize}
~\\


\subsubsection{Exemples}
Utilisation de base voir figure \ref{crossV}

~\\
\begin{figure}[H]
\begin{center}
\begin{verbatim}
$ perl execAndEval.pl --root "../../create_db/Eukaryota__2759" 
                      --start "100" 
                      --end "300" 
                      --step "50" 
                      --list list.txt 
                      --kmer "pattern.txt" 
                      --sample "20"
\end{verbatim}
\end{center}
\caption{\label{crossV}Utilisation du script execAndEval.pl}
\end{figure}
~\\

Si on lance la commande de la figure \ref{crossV} et si list.txt correspond au fichier de la figure \ref{fileList}
alors notre fichier aura la forme de celui présenté en figure \ref{resultlog}

~\\
\begin{figure}[H]
\begin{center}

\begin{verbatim}
$ cat list.txt
-1
1000
200
\end{verbatim}
\end{center}
\caption{\label{fileList}Contenu du fichier list.txt}
\end{figure}
~\\

\begin{figure}[H]
\begin{verbatim}
$ cat list.txt
complete    100	    61.7
complete	  150	    63.1
complete	  200	    64.7
complete	  250	    73.9
complete	  300	    77.3
1000	      100	    68.9
1000	      150	    44.3
1000	      200	    50.7
1000	      250	    66.9
1000	      300	    62.9
200	        100	    91.3
200	        150	    98.4
200	        200	    97.3
200	        250	    98.4
200	        300	    99.0

\end{verbatim}
\begin{center}

\end{center}
\caption{\label{resultlog}Contenu du fichier result.log}
\end{figure}
~\\



\subsection{Génération d'une courbe à partir d'un fichier tabulé}
\label{graph2}

\begin{verbatim}
  $ cd generate_graph
\end{verbatim}
\subsubsection{Prérequis}

Un fichier au format tabulé avec à la première colonne
la taille du read de l'espace d'apprentissage, à la suivante 
la taille du read de l'espace de prédiction et à la dernière colonne
au taux de vrais positifs

\subsubsection{Résultats attendus}

L'outil génère un fichier au format pdf représentant le graphe de vrai positif
en fonction de la taille du read. 


\subsubsection{Exemples}
Utilisation de base voir figure \ref{makgraph}

~\\
\begin{figure}[H]
\begin{center}
\begin{verbatim}
$ perl ./make_graph.pl -in ../cpp/count_kmer/result.log -title eukaryotaComplete
\end{verbatim}
\end{center}
\caption{\label{makgraph}Utilisation du script make\_graph.pl}
\end{figure}
~\\




\end{document}



