\subsection{Récupération du Genbank}
\label{qwery}
\subsubsection{Prérequis}

Dans l'espace de travail par défaut se trouve un fichier REQUIRED, qui un
script d'installation de paquet Debian nécessaire  pour l'utilisation des scripts perl, voir figure \ref{installPerl}

~\\
\begin{figure}[H]

\begin{center}
\begin{verbatim}
  $ ls -1
count_kmer/  cpp/  create_db/  data_mining/  docs/  filter/ generate_data/  generate_graph/  
generate_learn/  HOWTO/  qwery/ REQUIRED

  $ sh REQUIRED
\end{verbatim}


\end{center}
\caption{\label{installPerl}Installation des bibliothèque perl}
\end{figure}
~\\

On peut à présent se placer dans le dossier qwery pour effectuer notre requete.

\begin{verbatim}
  $ cd qwery/
  
  $ ls
  qweryNCBI.pl
\end{verbatim}

\subsubsection{Résultats attendus}

On s'attend ici à récupérer un fichier au format Genbank du ncbi\footnote{http://www.ncbi.nlm.nih.gov/} des génomes
mitochondriaux complet.

La requête exécutée par défaut si l'id = 2759 est : 
\begin{verbatim}
  txid2759[Organism:exp] AND (mitochondria[Title] OR mitochondrion[Title] OR mitochondrial[Title]) AND "complete genome"[Title]
\end{verbatim}

\subsubsection{Utilisation}
\paragraph{Les options}
\begin{itemize}
 \item Obligatoires :
\begin{itemize}
 
     \item id: taxid du taxon à récupérer ( -id 2759 )
     \item m: email de l'utilisateur ( -m myemail@mail.com )
    \item out: nom de sortie sans extension ( -out eukaryota )
  \end{itemize}
  \item Optionelles:
  \begin{itemize}
    \item path : dossier où sauvegarder le genbank ( /home/me/bdd/ )
    \item not : liste d'id à ne pas récupérer ( -not 33630,33258 )
    \item more : ajouter des précisions à la requêtes par défaut ( -more "NOT HOMO" )
    \item mine : spécifier sa propre requête ( -mine txid2759[Organism:exp] )
    \item help : Affiche l'aide.
  \end{itemize}
\end{itemize}


\subsubsection{Exemples}

Utilisation basique voir figure \ref{ub}:

~\\
\begin{figure}[H]

\begin{center}
\begin{verbatim}
  $ ls 
  qweryNCBI.pl
  
  $ ./qweryNCBI.pl -id 2759 -m toto@mail.com -out eukaryota
  
  $ ls
  qweryNCBI.pl eukaryota.gb
  \end{verbatim}
\end{center}
\caption{\label{ub}Utilisation minimale de qweryNCBI.pl}
\end{figure}
~\\

\subsection{Création des dossier}
\label{createdb}

\begin{verbatim}
  $ cd create_db
  $ ls 
  bdd/ Eukaryota_krona.html  generateDirectories.pl  
  get_dump_file.sh  get_leaf.sh  install_krona.sh
\end{verbatim}
\subsubsection{Prérequis}

Un fichier genbank est nécessaire pour la création de la base de donnée.
Les requêtes sont par défaut effectuées en locale il est donc d'avoir la structure arborescente du NCBI 
dans des fchiers plats, ces fichiers sont versionnés sur la forge et sont présent dans create\_db/bdd/, mais on peut également les récupérer ou mettre
à jour via le script\textit{ get\_dump\_file.sh }

\begin{verbatim}
  $ sh get_dump_file.sh
  \end{verbatim}



\subsubsection{Résultats attendus}
Le script produit une base de donnée avec une structure arborescente de dossier. Il produit également, la structure
de la base dans un format newick, un fichier xml au format adéquat pour l'outil \textit{krona}, et un script pour le nettoyage de la base.

Le script \textit{install\_krona.sh} est prévu pour l'installation de krona, si on souhaite l'utiliser voir figure \ref{krona}.

Enfin il va générer deux fichier dans le dossier generate\_data pour la génération
des données pour l'étape \ref{generatedata}.
~\\
\begin{figure}[H]

\begin{center}
\begin{verbatim}
14:47:09 [jeremy][create_db]$ ./install_krona.sh 

*********************
Debut install krona
*********************


Cloning into 'krona'...
remote: Counting objects: 788, done.
remote: Compressing objects: 100% (598/598), done.
remote: Total 788 (delta 364), reused 305 (delta 149)
Receiving objects: 100% (788/788), 528.43 KiB | 395 KiB/s, done.
Resolving deltas: 100% (364/364), done.
Creating links...

Installation complete.

To use scripts that rely on NCBI taxonomy, run updateTaxonomy.sh to build the
local taxonomy database.


**************************
Fin install krona, exemple
**************************


*********************
ktImportXML krona_Eukaryota.xml
*********************


14:47:19 [jeremy][create_db]$ 

  \end{verbatim}
\end{center}
\caption{\label{krona}Installation de krona}
\end{figure}
~\\

\subsubsection{Utilisation}
\paragraph{Les options}
\begin{itemize}
 \item Obligatoires :
\begin{itemize}
 
     \item id: taxid du taxon  ( -id 2759 )
     \item gen: fichier genbank ( -gen file.gb )
    \item bound: nombre de sequence pour creer un dossier ( -bound 10 )
  \end{itemize}
  \item Optionelles:
  \begin{itemize}
    \item path : dossier où créer la base de donnée ( /home/me/bdd/ )
    \item time : temps pris pour la création de la base ( -time )
    \item help : Affiche l'aide.
  \end{itemize}
\end{itemize}

\subsubsection{Exemples}

Utilsation de base voir figure \ref{gd}:

~\\
\begin{figure}[H]

\begin{center}
\begin{verbatim}
  $ ls
  bdd/  Eukaryota_krona.html  generateDirectories.pl  get_dump_file.sh  
  get_leaf.sh  install_krona.sh 
  
  $ ls ../generate_data 
  conf  extractGenbank.pl  fillAll_v2.sh
  
  $ ./generateDirectories.pl -id 33630 -gen ../qwery/alveolata.gb -bound 10
  
  $ ls 
  Alveolata__33630/  bdd/  Eukaryota_krona.html  generateDirectories.pl  get_dump_file.sh  
  get_leaf.sh  install_krona.sh  krona_Alveolata.xml  script_clean_Alveolata.sh  
  tree_Alveolata.newick
  
  $ ls ../generate_data 
  conf  extractGenbank.pl  fillAll_v2.sh	generateGenbank_Alveolata.sh  listGenbank.txt
  \end{verbatim}
\end{center}
\caption{\label{gd}Utilisation minimale de generateDirectories.pl}
\end{figure}
~\\



\subsection{Création des dossier}
\label{generatedata}
\begin{verbatim}
  $ cd generate_data
  $ ls
  conf  extractGenbank.pl  fillAll_v2.sh	generateGenbank_Alveolata.sh  listGenbank.txt
\end{verbatim}

\subsubsection{Prérequis}
Le fichier listGenbank, généré grâce au script présenté en \ref{createdb} est nécessaire 
pour la génération des données aux bons endroits. En effet ce fichier contient des listes d'accessions propre
à un dossier.

\subsubsection{Résultats attendus}
On s'attend, après avoir exécuté ce scripts, l'ensemble des données dans la base. C'est à dire
les genbanks correspondants, les génomes, les séquences codantes (selon le fichier conf)...

\subsubsection{Utilisation, exemple}

~\\
\begin{figure}[H]
\begin{center}
\begin{verbatim}
  $ sh generateGenbank_Alveolata.sh 
  \end{verbatim}
\end{center}
\caption{Génération des données}
\end{figure}
~\\


Attention, ce script produit un affichage sur stdout montrant l'évolution du script dans ces trois étapes:

\begin{verbatim}
      Traitement 1 / 3
      Avancement: 100 %
      Traitement 2 / 3 (Génération des donnees aux feuilles)
      Traitement 3/3 (linkage)
\end{verbatim}



