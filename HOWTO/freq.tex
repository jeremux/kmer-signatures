\subsection{Calcul de fréquence en c++}
\label{cpp}

  \begin{verbatim}
  $ cd cpp/count_kmer/
\end{verbatim}
Il existe également une version de calcul de fréquence de kmer en C, mais pour
le projet on a opté pour le c++ pour son côté objet. Les versions développés en Perl, Java et C
ont servi de test pour la version c++.

\subsubsection{Utilisation}
Les options
\begin{itemize}
  \item \textit{listFasta}           fichier contenant une liste de chemins vers des fichiers fasta,
		\item \textit{fasta }              fichier fasta,
		\item\textit{wsize}               taille du read,
		\item \textit{noData}              libérer la mémoire suite au chargement des données,
		\item \textit{kmer}                fichier contenant les patterns de kmers,
		\item \textit{output}              nom du fichier de sortie,
		\item \textit{version}             version du programme,
		\item \textit{test}                lance une batterie de tests unitaires,
		\item \textit{intra}               lance un test sur le taxon Intramacronucleata,
		\item \textit{key}                 sur quelles séquences compter: cox1,cox2,genomes...,
		\item \textit{root}                racine où on doit établir l’apprentissage,
		\item \textit{jump}                taille du décalage de la fenêtre,
		\item \textit{learn}               taille de l'apprentissage pour la validation croissée,
		\item \textit{start}               début de la taille du read dans la prédiction pour la validation croisée,
		\item \textit{end}                 fin de la taille du read dans la prédiction pour la validation croisée,
		\item \textit{step}                taille du pas pour les différentes fenêtre pour la prédiction,
		\item \textit{sample}              nombre de séquences à considérer par taxon,
		\item \textit{weka}                imprime sur la sortie le format weka,
		\item \textit{help}                affiche l'aide.
\end{itemize}
\subsubsection{Exemples}


~\\
\begin{figure}[H]

Pour produire un fichier de fréquence au format weka avec un fichier fasta voir la figure \ref{freq1}.
Cependant utiliser un fichier simple ne produira pas de classe à prédire pour cela il faut fournir un dossier \ref{freq2}
avec l'architecture établi par le programme présenté en \ref{createdb}.
\begin{center}
\begin{verbatim}
  $ make
  
  $ ./count_kmer -f seq.fasta 
                 -k pattern.txt 
                 -o out.arff 
                 -l -1 
                 --weka
  
  $ cat out.arff
  @RELATION freqKmer

  @ATTRIBUTE aaaa NUMERIC
  @ATTRIBUTE aaac NUMERIC
  @ATTRIBUTE aaag NUMERIC
  @ATTRIBUTE aaat NUMERIC
  ...
  @ATTRIBUTE tttg NUMERIC
  @ATTRIBUTE tttt NUMERIC
  @ATTRIBUTE class {null}

  @DATA
  0.25,0,...,null %Data(0,0)
  ...

  \end{verbatim}
\end{center}
\caption{\label{freq1}Produit un fichier weka avec count\_kmer}
\end{figure}

~\\
\begin{figure}[H]
\begin{center}
\begin{verbatim}
  $ makefile
  
  $ ./count_kmer -f seq.fasta 
                 -k pattern.txt 
                 -o out.arff 
                 -l -1 
                 --weka 
                 --root ~/trunk/create_db/Eukaryota__2759/Alveolata__33630
  
  $ cat out.arff
  @RELATION freqKmer

  @ATTRIBUTE aaaa NUMERIC
  @ATTRIBUTE aaac NUMERIC
  @ATTRIBUTE aaag NUMERIC
  @ATTRIBUTE aaat NUMERIC
  ...
  @ATTRIBUTE tttg NUMERIC
  @ATTRIBUTE tttt NUMERIC
  @ATTRIBUTE class {5794,5878,others}

  @DATA
  0.15,0.64,...,5878 %Data(0,0)

  \end{verbatim}
\end{center}
\caption{\label{freq2}Produit un fichier weka avec count\_kmer en précisant un dossier de taxon}
\end{figure}
~\\
